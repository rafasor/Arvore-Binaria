
\documentclass[a4paper,12pt]{article}
\usepackage[utf8]{inputenc}
\usepackage[brazil]{babel}
\usepackage{amsmath}
\usepackage{listings}
\usepackage{graphicx}
\usepackage{fancyhdr}
\usepackage{geometry}
\geometry{a4paper, margin=2.5cm}

\title{Trabalho Prático: Árvore Binária}
\author{Nome da Equipe \\ Centro Universitário UNA - Sete Lagoas}
\date{Abril de 2025}

\begin{document}

\maketitle
\tableofcontents
\newpage

\section{Introdução}

O presente trabalho tem como objetivo desenvolver um programa em Java que implemente uma árvore binária com suas principais funcionalidades. O sistema permite inserção e remoção de elementos, identificação de características da árvore e diferentes percursos. O objetivo é consolidar conhecimentos sobre estruturas de dados e programação orientada a objetos.

\section{Implementação}

\subsection{Estruturas de Dados}

\begin{lstlisting}[language=Java, caption=Classe Node]
public class Node {
    int valor;
    Node esquerda, direita;

    public Node(int valor) {
        this.valor = valor;
        esquerda = direita = null;
    }
}
\end{lstlisting}

\begin{lstlisting}[language=Java, caption=Classe BinaryTree]
public class BinaryTree {
    Node raiz;
    // Métodos: inserir, buscar, percursos
}
\end{lstlisting}

\subsection{Funcionalidades}

- Inserção e Remoção de nós
- Verificação de árvore cheia, completa ou estritamente binária
- Impressão dos percursos
- Grau de um nó
- Nível da árvore
- Busca de elemento

\section{Testes Executados}

\begin{tabular}{|l|c|c|}
\hline
\textbf{Operação} & \textbf{Entrada} & \textbf{Saída esperada} \\
\hline
Inserção & 10, 5, 20 & Árvore com 3 nós \\
Busca de elemento & 5 & Encontrado \\
Verificar se é completa & Após inserção & True \\
Grau do nó 10 & 10 & Grau = 2 \\
Nível da árvore & - & Nível = 2 \\
\hline
\end{tabular}

\section{Conclusão}

A implementação da árvore binária consolidou conceitos de estrutura de dados e OOP. As dificuldades enfrentadas incluíram a remoção e verificação das propriedades da árvore.

\section{Bibliografia}

\begin{itemize}
    \item Jovana, S. (2019). \textit{Referências bibliográficas da ABNT}. Acessado em 04/04/2025.
    \item Team, O. (2025). \textit{Overleaf}. Acessado em 04/04/2025.
\end{itemize}

\end{document}
